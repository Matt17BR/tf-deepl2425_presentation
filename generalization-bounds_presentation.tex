% Options for packages loaded elsewhere
\PassOptionsToPackage{unicode}{hyperref}
\PassOptionsToPackage{hyphens}{url}
%
\documentclass[
  ignorenonframetext,
]{beamer}
\usepackage{pgfpages}
\setbeamertemplate{caption}[numbered]
\setbeamertemplate{caption label separator}{: }
\setbeamercolor{caption name}{fg=normal text.fg}
\beamertemplatenavigationsymbolsempty
% Prevent slide breaks in the middle of a paragraph
\widowpenalties 1 10000
\raggedbottom
\setbeamertemplate{part page}{
  \centering
  \begin{beamercolorbox}[sep=16pt,center]{part title}
    \usebeamerfont{part title}\insertpart\par
  \end{beamercolorbox}
}
\setbeamertemplate{section page}{
  \centering
  \begin{beamercolorbox}[sep=12pt,center]{part title}
    \usebeamerfont{section title}\insertsection\par
  \end{beamercolorbox}
}
\setbeamertemplate{subsection page}{
  \centering
  \begin{beamercolorbox}[sep=8pt,center]{part title}
    \usebeamerfont{subsection title}\insertsubsection\par
  \end{beamercolorbox}
}
\AtBeginPart{
  \frame{\partpage}
}
\AtBeginSection{
  \ifbibliography
  \else
    \frame{\sectionpage}
  \fi
}
\AtBeginSubsection{
  \frame{\subsectionpage}
}

\usepackage{amsmath,amssymb}
\usepackage{iftex}
\ifPDFTeX
  \usepackage[T1]{fontenc}
  \usepackage[utf8]{inputenc}
  \usepackage{textcomp} % provide euro and other symbols
\else % if luatex or xetex
  \usepackage{unicode-math}
  \defaultfontfeatures{Scale=MatchLowercase}
  \defaultfontfeatures[\rmfamily]{Ligatures=TeX,Scale=1}
\fi
\usepackage{lmodern}
\usetheme[]{Dresden}
\usefonttheme{structurebold}
\ifPDFTeX\else  
    % xetex/luatex font selection
\fi
% Use upquote if available, for straight quotes in verbatim environments
\IfFileExists{upquote.sty}{\usepackage{upquote}}{}
\IfFileExists{microtype.sty}{% use microtype if available
  \usepackage[]{microtype}
  \UseMicrotypeSet[protrusion]{basicmath} % disable protrusion for tt fonts
}{}
\makeatletter
\@ifundefined{KOMAClassName}{% if non-KOMA class
  \IfFileExists{parskip.sty}{%
    \usepackage{parskip}
  }{% else
    \setlength{\parindent}{0pt}
    \setlength{\parskip}{6pt plus 2pt minus 1pt}}
}{% if KOMA class
  \KOMAoptions{parskip=half}}
\makeatother
\usepackage{xcolor}
\newif\ifbibliography
\setlength{\emergencystretch}{3em} % prevent overfull lines
\setcounter{secnumdepth}{-\maxdimen} % remove section numbering


\providecommand{\tightlist}{%
  \setlength{\itemsep}{0pt}\setlength{\parskip}{0pt}}\usepackage{longtable,booktabs,array}
\usepackage{calc} % for calculating minipage widths
\usepackage{caption}
% Make caption package work with longtable
\makeatletter
\def\fnum@table{\tablename~\thetable}
\makeatother
\usepackage{graphicx}
\makeatletter
\def\maxwidth{\ifdim\Gin@nat@width>\linewidth\linewidth\else\Gin@nat@width\fi}
\def\maxheight{\ifdim\Gin@nat@height>\textheight\textheight\else\Gin@nat@height\fi}
\makeatother
% Scale images if necessary, so that they will not overflow the page
% margins by default, and it is still possible to overwrite the defaults
% using explicit options in \includegraphics[width, height, ...]{}
\setkeys{Gin}{width=\maxwidth,height=\maxheight,keepaspectratio}
% Set default figure placement to htbp
\makeatletter
\def\fps@figure{htbp}
\makeatother

\AtBeginSection{}
\titlegraphic{\includegraphics[width=0.33\textwidth]{lmu-logo.png}}
\definecolor{lmugreen}{RGB}{0,136,58}
\setbeamercolor{section in toc}{fg=black,bg=white}
\setbeamercolor{alerted text}{fg=lmugreen!80!gray}
\setbeamercolor*{palette primary}{fg=lmugreen!60!black,bg=gray!30!white}
\setbeamercolor*{palette secondary}{fg=lmugreen!70!black,bg=gray!15!white}
\setbeamercolor*{palette tertiary}{bg=lmugreen!80!black,fg=gray!10!white}
\setbeamercolor*{palette quaternary}{fg=lmugreen,bg=gray!5!white}
\setbeamercolor*{sidebar}{fg=lmugreen,bg=gray!15!white}
\setbeamercolor*{palette sidebar primary}{fg=lmugreen!10!black}
\setbeamercolor*{palette sidebar secondary}{fg=white}
\setbeamercolor*{palette sidebar tertiary}{fg=lmugreen!50!black}
\setbeamercolor*{palette sidebar quaternary}{fg=gray!10!white}
\setbeamercolor{titlelike}{parent=palette primary,fg=lmugreen}
\setbeamercolor{frametitle}{bg=lmugreen!10}
\setbeamercolor{frametitle right}{bg=gray!60!white}
\setbeamercolor{title}{bg=lmugreen!10, fg=lmugreen}
\setbeamercolor{item}{fg=lmugreen}
\setbeamercolor{enumerate item}{fg=lmugreen}
\setbeamercolor*{separation line}{}
\setbeamercolor*{fine separation line}{}
\setbeamertemplate{footline}%{miniframes theme}
{%
  \begin{beamercolorbox}[colsep=1.5pt]{upper separation line foot}
  \end{beamercolorbox}
  \begin{beamercolorbox}[ht=2.5ex,dp=1.125ex,%
    leftskip=.3cm,rightskip=.3cm plus1fil]{author in head/foot}%
    \leavevmode{\usebeamerfont{author in head/foot}\insertshortauthor}%
    \hfill%
    {\usebeamerfont{institute in head/foot}\usebeamercolor[fg]{institute in head/foot}\insertshortinstitute}%
  \end{beamercolorbox}%
  \begin{beamercolorbox}[ht=2.5ex,dp=1.125ex,%
    leftskip=.3cm,rightskip=.3cm plus1fil]{title in head/foot}%
    {\usebeamerfont{title in head/foot}\insertshorttitle} \hfill     \insertframenumber/\inserttotalframenumber%
  \end{beamercolorbox}%
  \begin{beamercolorbox}[colsep=1.5pt]{lower separation line foot}
  \end{beamercolorbox}
}
\makeatletter
\@ifpackageloaded{caption}{}{\usepackage{caption}}
\AtBeginDocument{%
\ifdefined\contentsname
  \renewcommand*\contentsname{Table of contents}
\else
  \newcommand\contentsname{Table of contents}
\fi
\ifdefined\listfigurename
  \renewcommand*\listfigurename{List of Figures}
\else
  \newcommand\listfigurename{List of Figures}
\fi
\ifdefined\listtablename
  \renewcommand*\listtablename{List of Tables}
\else
  \newcommand\listtablename{List of Tables}
\fi
\ifdefined\figurename
  \renewcommand*\figurename{Figure}
\else
  \newcommand\figurename{Figure}
\fi
\ifdefined\tablename
  \renewcommand*\tablename{Table}
\else
  \newcommand\tablename{Table}
\fi
}
\@ifpackageloaded{float}{}{\usepackage{float}}
\floatstyle{ruled}
\@ifundefined{c@chapter}{\newfloat{codelisting}{h}{lop}}{\newfloat{codelisting}{h}{lop}[chapter]}
\floatname{codelisting}{Listing}
\newcommand*\listoflistings{\listof{codelisting}{List of Listings}}
\makeatother
\makeatletter
\makeatother
\makeatletter
\@ifpackageloaded{caption}{}{\usepackage{caption}}
\@ifpackageloaded{subcaption}{}{\usepackage{subcaption}}
\makeatother

\ifLuaTeX
  \usepackage{selnolig}  % disable illegal ligatures
\fi
\usepackage{bookmark}

\IfFileExists{xurl.sty}{\usepackage{xurl}}{} % add URL line breaks if available
\urlstyle{same} % disable monospaced font for URLs
\hypersetup{
  pdftitle={Generalization Bounds},
  pdfauthor={Matteo Mazzarelli},
  hidelinks,
  pdfcreator={LaTeX via pandoc}}


\title{Generalization Bounds}
\subtitle{Theoretical Foundations of Deep Learning}
\author{Matteo Mazzarelli}
\date{December 17, 2024}

\begin{document}
\frame{\titlepage}


\section{Introduction}\label{introduction}

\begin{frame}{Motivation}
\phantomsection\label{motivation}
\begin{itemize}
\tightlist
\item
  \textbf{Core Challenge}: How can a model learned from \emph{limited
  training data} perform well on \emph{unseen data}?
\item
  Generalization lies at the heart of the machine learning process.
\item
  A poorly generalized model risks:

  \begin{itemize}
  \tightlist
  \item
    \textbf{Overfitting}: Performing well on training data but poorly on
    unseen data.
  \item
    \textbf{Underfitting}: Failing to capture the underlying patterns of
    the data.
  \end{itemize}
\end{itemize}
\end{frame}

\begin{frame}{The Perils of Overfitting: A Motivating Visualization}
\phantomsection\label{the-perils-of-overfitting-a-motivating-visualization}
\begin{itemize}
\tightlist
\item
  \textbf{Overfitting in Action}:

  \begin{itemize}
  \tightlist
  \item
    A model can perfectly fit training data but fail to capture the true
    underlying pattern.
  \item
    This often leads to poor performance on unseen data.
  \end{itemize}
\item
  \textbf{Demonstration}:

  \begin{itemize}
  \tightlist
  \item
    Dataset: A simple linear trend with noise.
  \item
    Models:

    \begin{itemize}
    \tightlist
    \item
      Linear model: Captures the underlying trend.
    \item
      High-degree polynomial: Overfits the noise in the data.
    \end{itemize}
  \end{itemize}
\end{itemize}
\end{frame}

\begin{frame}
\includegraphics{generalization-bounds_presentation_files/figure-beamer/overfitting-plot-1.pdf}
\end{frame}

\begin{frame}{The Learning Problem}
\phantomsection\label{the-learning-problem}
\begin{itemize}
\tightlist
\item
  \textbf{Supervised Learning}:

  \begin{itemize}
  \tightlist
  \item
    Goal: Learn a function ( f: X \to Y ) mapping inputs ( X ) to
    outputs ( Y ) based on labeled training data.
  \end{itemize}
\item
  \textbf{Key Question}: Can the learned function perform well on unseen
  data?
\item
  \textbf{Generalization}:

  \begin{itemize}
  \tightlist
  \item
    Ability of a model to extend its learning beyond the training data.
  \item
    \textbf{Central Problem} in machine learning: balancing
    \emph{empirical performance} with \emph{future predictions}.
  \end{itemize}
\end{itemize}
\end{frame}

\begin{frame}{Why Theory Matters}
\phantomsection\label{why-theory-matters}
\begin{itemize}
\tightlist
\item
  \textbf{Significance of Theory}:

  \begin{itemize}
  \tightlist
  \item
    Guides \textbf{algorithm design} by providing a foundation for
    developing new methods.
  \item
    Allows \textbf{performance analysis} to identify the strengths and
    weaknesses of algorithms.
  \item
    Reveals \textbf{limitations} of learning systems, helping us
    understand their boundaries.
  \end{itemize}
\item
  \textbf{Theoretical Understanding}:

  \begin{itemize}
  \tightlist
  \item
    Bridges the gap between empirical performance and guarantees on
    future behavior.
  \end{itemize}
\end{itemize}
\end{frame}

\begin{frame}{Introducing Generalization Bounds}
\phantomsection\label{introducing-generalization-bounds}
\begin{itemize}
\tightlist
\item
  \textbf{What Are Generalization Bounds?}

  \begin{itemize}
  \tightlist
  \item
    Theoretical tools offering guarantees about a model's performance on
    unseen data.
  \item
    Relate:

    \begin{itemize}
    \tightlist
    \item
      \textbf{Generalization Error}: How well the model generalizes.
    \item
      \textbf{Empirical Risk}: Performance observed on training data.
    \item
      \textbf{Model Complexity}: How expressive the model is.
    \end{itemize}
  \end{itemize}
\item
  \textbf{Purpose}:

  \begin{itemize}
  \tightlist
  \item
    Provide insights into the trade-offs between model accuracy,
    complexity, and training data size.
  \end{itemize}
\end{itemize}
\end{frame}




\end{document}
